%! Author = Dean
%! Date = 12/16/2023
\chapter{Neuronske mreže}\label{ch:neuronske-mreze}

Neuronske mreže čine podskup strojnog učenja i predstavljaju temelj algoritama dubokog učenja.
Inspiracija za strukturu i način rada neuronskih mreža crpi se iz ljudskog mozga pokušavajući oponašazi biološke neurone i njihovu međusobnu komunikaciju.


\section{Umjetni neuron}
Prvi model umjetnog neurona je razvijen od strane Warrena McCullocha i Waltera Pittsa, poznat kao McCulloch-Pitts
model umjetnog neurona. Ovaj model oponaša funkcionalnost biološkog neurona, gdje ulazni signali utječu na odluku neurona o tome hoće li
se on aktivirati ili ne
McCulloch-Pitts model umjetnog neurona sastoji se od \textit{n} ulaznih značajki ili signala, označenih s $x_1, x_2, \dots, x_n$, te njihovim pripadajućim težinama $w_1, w_2, \dots, w_n$ koje određuju važnost pojedinih informacija.
\FloatBarrier
\begin{figure}[h]
    \centering
    \includegraphics[width=0.8\textwidth]{images/Umjetni_neuron}
    \caption{McCulloch-Pitts model umjetnog neurona}
    \label{fig:slika1}
\end{figure}
\FloatBarrier
Težinska suma neurona \emph{net} se izračunava kao suma ulaznih značajki
\[ net = x_0*w_0 + x_1*w_1 + \dots + x_n*w_n \]
Gdje nam $w_0$ označava takozvana vrijednost praga \emph(bias) koji služi kako bismo odredili kada će neuron biti aktivan ili ne.
Dok nam $x_0$ služi samo radi lakšeg matematičkog zapisa i on je uvijek jedank 1. Znajući sve ovo sad možemo težinsku sumu napisati kao
\[ net = \sum_{i=0}^{\n} w_0*x_0 \]
Drugi način težinske sume koristi vekotorski oblik
\[ net =\vec{w}^T \cdot \vec{x} \]
Na kraju \textit{net} propustimo kroz takozvanu aktivacijsku funckiju \textit{f} koja će nam dati izlaz.

\subsection{Aktivacijska funkcija}
Aktivacijska funkcija je funkcija koja nam daje izlaz neurona na temelju težinske sume.
Glavna uloga aktivacijske funkcije je ta što ona odlučuje hoće li neuron biti aktivan ili ne ili bolje rečeno hoće neuron u toj situaciji biti
važan u ulozi predikcije.
Najjednostavnija moguća aktivacijska funkcija je funkcija identiteta
\[ f(\textit{net}) = net \]
Danas najčešće korištene aktivacijske funkcije us sigmoidalna u ReLU.
Sigmoidalna funkcija je definirana kao:
\[ f(\textit{net}) = \frac{1}{1 + e^-net} \]
\FloatBarrier
\begin{figure}[h]
    \centering
    \includegraphics[width=0.8\textwidth]{images/Sigmoid}
    \caption{Sigmoidalna funkcija}
    \label{fig:slika2}
\end{figure}
Kodomena sigmoidalne funkcije je (0, 1) gdje će za jako velike vrijednosti težiti ka 1 a za jako male vrijednosti težiti ka 0.
ReLU aktivacijska funkcija je definirana kao:
\[ f(\textit{net}) = \max(0, \textit{net}) \]
\FloatBarrier
\begin{figure}[h]
    \centering
    \includegraphics[width=0.8\textwidth]{images/ReLU}
    \caption{ReLU}
    \label{fig:slika3}
\end{figure}
\FloatBarrier

\section{Višeslojne neuronske mreže}
\subsection{Struktura mreže}
Način na koji su neuroni međusobno povezani i organizirani u mreži određuju njezinu arhitekturu te razlikujemo 4 osnovne arhitekture:

\begin{enumerate}
    \item aciklička mreža
    \item mreža s povratnom vezom
    \item lateralno povezana mreža
    \item hibridne mreže
\end{enumerate}

AcIklička mreža nema povratnih veza između neurona tako da je propagacija signala jednosmjerna.
Kod ovakvih vrsta mreža razlikujemo ulazni sloj neurona, skriveni sloj neurona i izlazni sloj neurona.
Ulazni sloj neurona nemaju ulazne signale odnosno nemaju ulogu neurona već samo propagiraju signal iz ulaznog sloja u prvi skriveni sloj.
Skriveni sloj se sastoji od jednog ili više sloja neurona. Skriveni slojevi igraju ključnu ulogu u procesu učenja i ekstrakciji značajki iz podataka.
I na kraju imamo izlazni sloj koji sadrži rezultat.
\FloatBarrier
\begin{figure}[h]
    \centering
    \includegraphics[width=0.8\textwidth]{images/nn-arch}
    \caption{Aciklička mreža}
    \label{fig:slika4}
\end{figure}
\FloatBarrier

Neuronske mreže s povratnom vezom sadrže u svojoj strukturi barem jednu povratnu vezu.
Što znači da postoji barem 1 ili više čvorova za koje ako prazimo izlaz neurona kroz sve moguće puteve, nakon konačnog broja koraka ćemo ponovno obići prvobitni čvor.
Kod ovakve arhitekture ne dijelimo slojeve na ulazne i izlane kao kod acikličke mreže već govorimo o vidljivim i skrivenim čvorovima.

\FloatBarrier
\begin{figure}[h]
    \centering
    \includegraphics[width=0.8\textwidth]{images/nn-povratna-veza}
    \caption{Lateralno povezana mreža}
    \label{fig:slika5}
\end{figure}
\FloatBarrier
\pagebreak
Lateralno povezane mreže predstavljaju neuronske mreže gdje neuroni u istom sloju međusobno komuniciraju.
Često se koristi kako bi se omogućila suradnja između neurona kako bi se pojačale određene značajke unutar sloja.
\FloatBarrier
\begin{figure}[h]
    \centering
    \includegraphics[width=0.8\textwidth]{images/Lateral-connected-nn}
    \caption{Lateralno povezana mreža}
    \label{fig:slika6}
\end{figure}
\FloatBarrier