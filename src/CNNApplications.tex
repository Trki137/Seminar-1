%! Author = Dean
%! Date = 12/30/2023

\chapter{Primjena konvolucijskih neuronskih mreža}\label{ch:primjena-konvolucijskih-neuronskih-mreza}

Konvolucijske neuronske mreže koriste se u području računalnog vida i prepoznavanja slika.
Glavna zaduženja u tom procesu obuhvaćaju prepoznavanje objekata, grupiranje objekata te interpretaciju onoga što slika prikazuje.

Osim toga, konvolucijske neuronske mreže često se primjenjuju u algoritmima preporuka.
Primjerice, na web trgovinama možemo vidjeti preporuke proizvoda koji bi nas mogli zanimati, temeljene na našem prethodnom ponašanju na platformi.

Jedno od najpopularnijih područja primjene konvolucijskih mreža danas je prepoznavanje lica.
To se široko koristi na socijalnim mrežama, u nadzornim sustavima, ali i u svrhu sigurnosti jer se temelji na identifikaciji korisnika putem prepoznavanja lica.
Socijalne mreže često koriste tehnologiju prepoznavanja lica kako bi olakšale označavanje ljudi na fotografijama ili primijenile različite filtre za transformaciju lica u kreativne i zabavne sadržaje.

No, primjena konvolucijskih mreža nije ograničena na zabavu i oglašavanje.
U medicini se također široko koriste, posebno u detekciji anomalija na magnetskim rezonancijskim slikama.

Također, konvolucijske neuronske mreže koriste se u autonomnim vozilima za prepoznavanje pješaka, vozila i drugih objekata na cestama.
Ova tehnologija ima ključnu ulogu u razvoju sigurnijih i učinkovitijih sustava samovozećih vozila.

U području znanstvenih istraživanja, konvolucijske mreže primjenjuju se za analizu kompleksnih podataka, poput astrofotografija, bioloških uzoraka i drugih znanstvenih slika.
Ova primjena omogućuje automatiziranu analizu i izvlačenje značajki iz ogromnih skupova podataka.

Ove primjene ilustriraju široki spektar mogućnosti konvolucijskih neuronskih mreža u različitim sektorima, pokazujći njihovu snažnu ulogu u analizi i interpretaciji vizualnih podataka.